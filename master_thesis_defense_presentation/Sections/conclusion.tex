\documentclass[../main.tex]{subfiles}
\graphicspath{{\subfix{../Figures/}}}
\begin{document}
    \begin{frame}{Discussion about leaking}         
        Leaking is indeed still an open problem in hypergraphs. In particular, it is indeed possible to prove that 
        \begin{block}{Leaking starting from the stationary distribution on a set $S$}
			$\chi_S^T(\prod_{t'\leq t}(D_S M_{t'})) \psi_S \geq 1 - t\phi(S)$
        \end{block}
    
    	But, in contrast, it is not possible to prove it when starting from a single vertex:
    	
    	\begin{block}{Leaking starting from centered distribution in one vertex}
			When $S^g = \{v\in S: \chi_S^T (\prod_{t'\leq t}(D_S M_{t'})) \chi_v \geq 1-\phi(S) t\}$ then $\text{vol}(S^g) \ngeq \frac{1}{2}\text{vol}(H)$ 
    	\end{block}
    \end{frame}
	
	\begin{frame}{Hypothesis for leaking}
		Mildly speaking, when averaging (starting vertex $\psi_S$) we count every crossing hyperedge once. Instead, when starting from a single vertex $\chi_v$, we count every crossing edge up to $\approx r$ times (in case the graph is $r$-uniform). This introduces an additional $r$ factor in the leaking $p_t(S) \geq 1-r\phi t$, but still it is not possible to prove that this claim holds for $t>dt$	\end{frame}
\end{document}