\documentclass[../main.tex]{subfiles}
\graphicspath{{\subfix{../Figures/}}}
\begin{document}
    \begin{frame}{Is this result as powerful as the one for graphs?} 
    	Although the mixing theorem found is the same as for graphs:
    	
		\begin{block}{Mixing in irregular hypergraphs}
			$I_t(k) - \pi(S_j(\bold{p}_{t})) \leq \sqrt{k} e^{-\frac{t\hat{\phi}^2}{4}}$
		\end{block}
	
		In practice, for hypergraphs this result could be problematic, in fact $k$ can be as large as $O(n2^n)$, which means that the conductance approximation factor is $\hat{\phi} \leq \sqrt{n\phi^*}$ rather than $\leq \sqrt{\log(n)\phi^*}$.    
	\end{frame}	

	\begin{frame}{Conductance is $O(1)$, but mixing is $O(n)$}
		\begin{lemma}
			There exists a multigraph s.t. the conductance is $\frac{1}{2}$ but the mixing time is $O(n)$.
		\end{lemma}
		Here is an example: 
		\begin{itemize}
			\item Nodes $[1,n]$ in a straight path.
			\item Edge $(i,i+1)$ has weight $w(i,i+1) = \sum_{j=0}^{i} w(i,i+1)$ so that weight doubles at every $i$.
			\item Total graph volume: $2^{n-1}$.
			\item Conductance is $\frac{1}{2}$.
			\item Mixing time is $O(n)$ for a large fraction of nodes.
		\end{itemize}
		Can there be a hypergraph which gets collapsed into such an unfavourable multigraph?
	\end{frame}

	\begin{frame}{An argument for not having this issue in hypergraphs}
		Recall that the actual mixing theorem found with our analysis for the discrete diffusion process is
		\begin{block}{Improved mixing theorem for irregular hypergraphs}
			$I_t(k) - \pi(S_j(\bold{p}_t)) \leq \sqrt{\frac{k}{d(v_0)}}e^{-\frac{\hat{\phi}t}{
			4}}$
		\end{block}
		This means that when the volume of the hypergraph is exponential, but also the degree of the starting vertex is exponential, then we still mix in logarithmic time. But, if the volume of the hypergraph is exponential, then the probability of picking as starting node a node with not exponential degree is basically zero.
	\end{frame}
\end{document}