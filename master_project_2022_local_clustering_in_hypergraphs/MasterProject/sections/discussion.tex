\documentclass[../main.tex]{subfiles}
\graphicspath{{\subfix{../img/}}}
\begin{document}


    
\subsection{Is this result enough for logarithmic mixing time?}
\label{subsec:discussion_sublinear_mixing_time}

In this section, we are going to discuss the mixing result found for non $d$-regular hypergraphs (Theorem \ref{theorem:mixing_theorem}) and try to understand if it is enough to achieve a mixing time $O\left(\frac{\log(n)}{\phi^2}\right)$ as in the case of regular hypergraphs.

\subsubsection{Difference with d-regular mixing result}

To begin with, we are going to state back to back the two mixing results: first the one for non $d$-regular hypergraphs:

\begin{equation}
    I_t(k) \leq \sqrt{\hat{k}} e^{-\frac{\phi^2 t}{4}} + \frac{k}{\text{vol}(G)}
\end{equation}

And second, the one for $d$-regular hypergraphs:

\begin{equation}
    I_t(k) \leq \sqrt{\frac{\hat{k}}{d}} e^{-\frac{\phi^2 t}{4}} + \frac{k}{\text{vol}(G)}
\end{equation}

Recall that $\hat{k}$ can be as large as $\frac{1}{2}\text{vol}(H)$. The only clear difference among the two results is the $d$ factor: the $d$ regularity factor, in fact, is not a thing for non $d$-regular hypergraphs, and hence it is perfectly reasonable to think that it somehow disappears in the generalized version of the theorem. Moreover, in the Lovasz-Simonovits mixing result for standard non $d$-regular graphs, the convergence result is perfectly equivalent to our generalization to non $d$-regular hypergraphs. What we may ask is: does this difference affect the quality of the result?

If we limit our analysis to standard graphs, we can see that the bound $I_t(k) \leq \sqrt{\hat{k}} e^{-\frac{\phi^2 t}{4}} + \frac{k}{\text{vol}(G)}$ is perfectly fine for mixing time in $O\left(\frac{\log(n)}{\phi^2}\right)$: in fact, the volume of $G$ (and hence $\hat{k})$ can be only as large as ${n \choose 2} \leq n^2$. So, if we want to limit the error of the Lovasz-Simonovits curve to any number $O\left(\frac{1}{\text{poly}(n)}\right)$, it is sufficient to have a number of iterations:

\begin{align}
    \sqrt{\hat{k}} e^{\frac{-t \phi^2}{4}} \leq 
    \sqrt{n^2} e^{\frac{-t \phi^2}{4}}
    \leq \text{poly}\left(\frac{1}{n}\right) &\implies 
    e^{-\frac{t \phi^2}{4}} \leq \frac{1}{n^2} \\
    & \implies t =  O\left(\frac{\log(n)}{\phi^2}\right)
\end{align}

But if we turn our attention to hypergraphs, then the volume of the hypergraph (and also of the collapsed multigraph) can be as large as $n 2^{n-1}$ (when all hyperedges are present, every node appears in exactly $2^{n-1}$ hyperedges). This yields a mixing time of:

\begin{align}
    \sqrt{\hat{k}} e^{\frac{-t \phi^2}{4}} \leq 
    \sqrt{n2^n} e^{\frac{-t \phi^2}{4}}
    \leq \text{poly}\left(\frac{1}{n}\right) &\implies 
    e^{-\frac{t \phi^2}{4}} \leq \frac{1}{n\sqrt{n2^n}} \\
    & \implies t = O\left(\frac{\log(n2^n) + \log(n)}{\phi^2}\right) = O\left(\frac{n}{\phi^2}\right)
\end{align}

which is now way worse than the result found for graphs.

Notice that, instead, the mixing result for $d$-regular hypergraphs has the same logarithmic complexity as the standard graph mixing result: in fact, regardless of the total volume of the graph (which, recall, can be as large as $O(2^n)$), the following relationship binds the volume of the graph and the degree in any $d$-regular hypergraph:

\begin{equation}
    \text{vol}(H) = n d
\end{equation}

This claim ensures that the convergence result of Theorem \ref{theorem:mixing_theorem_regular_hypergraphs}

\begin{equation}
    I_t(k) \leq \sqrt{\frac{\hat{k}}{d}} e^{-\frac{t\phi^2}{4}} + \frac{k}{\text{vol}(G)}
\end{equation}

is enough to have $O\left(\frac{\log(n)}{\phi^2}\right)$ mixing time: in fact, 

$\sqrt{\frac{\hat{k}}{d}} \leq \sqrt{\frac{\text{vol}(G)}{d}} \leq \sqrt{n}$, so that with $t = O\left(\frac{\log(n)}{\phi^2}\right)$ you can mix with an error up to $\frac{1}{\text{poly}(n)}$.

This little difference in the mixing result has some undesired and unexpected consequences: for standard graphs, the main parameter that affects the mixing time is the conductance. In fact, when the conductance is large (like a constant), then you can expect to mix very fast in $t=O(\log(n))$. Whenever instead the conductance is low (like a path graph with conductance $O\left(\frac{1}{n}\right)$), then you cannot hope to mix in a time which is smaller than $n$ (it is, indeed, $t=O(\log(n)n^2)$). Although this principle applies for $d$-regular hypergraphs as well, at least by looking at Theorem \ref{theorem:mixing_theorem} it is not possible to conclude the same also for irregular hypergraphs: what the theorem suggests is that it might happen that a hypergraph has very large conductance ($\phi=O(1)$), but still a mixing time $O(n)$ because the volume of the hypergraph is extremely large. 

The main problem about having such a high mixing time for a clustering algorithm, is that the guarantee on the output conductance inevitably gets worse: in fact, if having a mixing time $O\left(\frac{\log(n)}{\phi^2}\right)$ ensures that the output conductance is at most $\leq \sqrt{\log(n) \phi^*}$ (with $\phi^*$ the optimal conductance), then having a mixing time $O\left(\frac{n}{\phi^2}\right)$ allows just a poor $\leq \sqrt{n \phi^*}$ approximation, which is extremely worse than the approximation for graphs and regular hypergraphs when $n$ gets large.

In the following sections we are going to discover whether this is actually the case: can there be a hypergraph with high conductance and large volume, such that even if the conductance is a large constant, the mixing time is still $O(n)$?

We start with a simple observation: notice that Theorem \ref{theorem:mixing_theorem} manages to find a bound on the mixing time using the collapsed multigraphs $G_t$'s. Hence, in order to prove that it is impossible to mix in time $O(n)$ when the conductance is a large constant, it might be tempting to prove such claim in multigraphs (which are easier to handle than hypergraphs).

In fact, if we can say that no such multigraph with high conductance and exponential volume exists, then it is also not possible to collapse a hypergraph into a multigraph with such undesirable characteristics.

Unfortunately, we will show in next Section \ref{subsubsec:counter_example} that such a multigraph with high conductance and exponential volume indeed exists.

\subsubsection{An example of a multigraph with high conductance and O(n) mixing time.}
\label{subsubsec:counter_example}

In this section, we are going to provide an example of a multigraph that has high constant conductance $\frac{1}{2}$, but still has a mixing time $\geq n$ for some special starting probability vectors $\vec{p}_0$. 
Here is the example: assume we have $n$ vertices like in a path, but allowing repetitions of the edges. In particular, $v_1$ (the leftmost node) has degree 1 and is connected to $v_2$. In turns $v_2$ has an outgoing edge with the same weight as all the edges coming before it (in this case, only 1). $v_3$ then has one incoming edge from $v_2$, and one outgoing edge going to $v_4$ with weight 2 (the sum of the edge weights coming before it).

In particular, the following rule applies in order to describe the weight between two consecutive vertices $a(i, i+1)$:

\begin{equation}
    a(i, i+1) = \sum_{j=1}^{i-1} a(j, j+1)
\end{equation}

namely, the weight of the edge going from $i$ to $i+1$ is the sum of all weights of the edges coming before $i$ (including the weight of the edge $(i-1, i)$). I.e the weights of the edges are: 1, 1, 2, 4, 8, 16 ...  

We will address this special graph as the \textit{path graph with exponential edge weights}.

We are going to prove this fact regarding the path graph with exponential edge weights:

\begin{claim}
\label{claim:path_graph_high_mixing_time}
    If $G$ is a path graph with exponential edge weights, then the conductance is constant ($\frac{1}{2}$), but the mixing time is $O(n)$.
\end{claim}

In order to bound the mixing time, we can see that the total time to reach $v_n$, when starting from vertex $v_1$, is $n$. Moreover, the last vertex has a stationary probability which is approximately $\frac{1}{4}$ (in fact, it is incident to half the edges in $G$, and the total volume is $2|E|$). So, if we start in $v_1$ with probability $p_0(1) = 1$, it is impossible to carry a constant amount of probability to a vertex which is at distance $n$, in less then $n$ steps. At the same time, we will show that the total volume is indeed $2^n$-ish and that the conductance is a constant, $\frac{1}{2}$.

Let's first prove that the volume of the graph is $2^n$: in particular, the volume of the graph is simply twice the sum of the weights of the edges:

\begin{align}
    \text{vol}(G) &= 2 \sum_{i=1}^{n-1} a(i, i+1) \\
    & = 2 \left(1 + \sum_{i=2}^{n-1} 2^{i-2}\right) \\
    & = 2 \left(1 + \sum_{i=0}^{n-3} 2^i \right) \\ 
    & = 2 \left(1 + \frac{1 - 2^{n-2}}{1 - 2} \right) \\
    & = 2^{n-1}
\end{align}

It is now easy to see that the conductance of every sweep cut $S_k = {v_1, v_2, ..., v_k}$ is $\frac{1}{2}$. The reason why is because the edge $(v_k, v_{k+1})$, which is the only edge that is crossing the cut, has weight exactly equal to $\sum_{i=1}^{k-1} a(i, i+1)$ by construction, which means that the volume of $S_k = \sum_{i=1}^{k} a(i, i+1) = \sum_{i=1}^{k-1} a(i, i+1) + a(k, k+1) = 2 a(k, k+1)$, hence only twice the volume of the cut. Now it remains to prove that no other cut can have a conductance which is lower than $\frac{1}{2}$.

\begin{lemma}
    In the path graph $G$ with increasingly exponential edge weights (as described in the above paragraph), no cut has a conductance lower than the sweep cuts.
\end{lemma}

The above lemma has the following direct corollary:

\begin{corollary}
    The path graph $G$ has conductance $\frac{1}{2}$.
\end{corollary}

\begin{proof}
    We will proceed by contradiction. In particular, assume that the cut is made of $k>1$ edges, like $e_{\pi_1}, e_{\pi_{2}}..., e_{\pi_{k}}$ and such that $e_{\pi_i}$ is an edge of the form $(v_j, v_{j+1})$ for some $j$. Assume that $\pi_{k}$ is the edge with highest index $j^*$. Then, the following scenario can happen: if $j^* = n-1$ (namely, we are cutting the last edge of weight $2^{n-3}$), then clearly the smallest bipartition must be smaller than half the volume of the entire graph, namely $\min(vol(S), \text{vol}(V\setminus S)) \leq 2^{n-2}$, and hence the conductance must be $\geq \frac{1}{2}$. Notice that the last edge  has been treated carefully because it might have been a bit tricky: the last node does not have the highest degree. The highest degree node is, in fact, the penultimate one. When, instead, $j^*$ is not $n-1$, then clearly the largest bipartition is the one on the right (because it contains the last two nodes that have volume $> \frac{1}{2} \text{vol}(G)$). Hence, given this edge cut, the highest conductance value we can achieve for this situation is by not partitioning further the left bipartition (because it would increase the number of edges crossing, and reduce the value of the volume of the left bipartition). But we have already proved that the conductance of every cut of one edge is $\frac{1}{2}$, hence we are done.
\end{proof}

This example, unfortunately, is saying that we cannot hope to mix in logarithmic time in any multigraph with high conductance. Hence, if we collapse our input hypergraph into a multigraph with the characteristics of the path graph just described, we might have a slow mixing time, even though the conductance is high.

What is left to be proved, is than that it is actually impossible to collapse a hypergraph into a multigraph with very high conductance and high volume: in the next Section we are actually going to prove that there is no hypergraph with high conductance (constant) and high volume (exponential). Hence, this is going to allow us to conclude that the mixing time also for irregular hypergraphs is mainly determined by the conductance, as it is the case for general graphs.

\subsection{Some solutions}
\label{subsec:some_solutions}

In this section we are going to discuss when the mixing result found for irregular hypergraphs can still be useful: in particular, in Section \ref{subsec:avg_r_related_to_conductance} we are going to prove that the mixing theorem found for irregular hypergraphs is actually almost equivalent to the one for graphs, when the conductance is high $\gg \frac{1}{n}$: in fact, we will show that there cannot be any hypergraph with high conductance and exponential volume, which means that as long as the conductance is high $\gg \frac{1}{n}$, then we are certain to mix fast $O\left(\frac{\log(n)}{\phi^3}\right)$ even in irregular hypergraphs.

Moreover, in Section \ref{subsec:choose_starting_vertex_random} we show that the mixing time is tightly related to the degree of the starting vertex: hence, when we can choose the starting vertex at random according to the stationary distribution, as in the case for most probabilistic clustering algorithms (\cite{SpielmanClustering}), then we can expect to have a fast mixing time $\frac{\log(n)}{\phi^2}$ rather than the mixing time showed by Theorem \ref{theorem:mixing_theorem}.

\subsubsection{High conductance implies low volume}
\label{subsec:avg_r_related_to_conductance}

In this section we are going to prove that if a hypergraph has high conductance, then it must have low volume ($\ll 2^n$): assume that the graph has high conductance $\phi \gg \frac{1}{n}$. Then, we are going to discuss that the average size of the edge cannot be too large: in particular, if we call $\hat{r}$ the average size of the hyperedges, then the following bound must hold: $\hat{r} \leq \frac{3}{\phi}$. Notice that it is the same bound as in the case of $r$-uniform hypergraphs, but with $\hat{r} := \frac{1}{m}\sum_{i=1}^{m} |e_i|$, namely the average hyperedge size. This allows us to conclude that the volume of the hypergraph cannot be larger than a poly factor of $n$ which gets handled by the logarithmic mixing time in Theorem \ref{theorem:mixing_theorem}.

We will prove the following lemma (very similar to the $r$-uniform hypergraph case)

\begin{lemma}
\label{lemma:convergence_with_avg_r_hypergraph}
    If the hypergraph $H$ (with no self loops) has high conductance $\phi \gg \text{poly}\left(\frac{1}{n}\right)$, the mixing time is low: $t = O\left(\frac{\log(n)}{\phi^3}\right)$.
\end{lemma}

We will prove this with the following argument: first, we will say that the average hyperedge size and the conductance are related.

\begin{lemma}
\label{lemma:hyperedge_avg_size_vs_phi}
    If the hypergraph (with no self loops) has conductance $\phi$, then the average size of the hyperedge $\hat{r}:=\frac{1}{m}\sum_{i=1}^{m} |e_i|$ is s.t. $\hat{r} \leq \frac{3}{\phi}$.
\end{lemma}

With this lemma at hand, we can say that when $\phi \gg \frac{1}{n}$ then

\begin{lemma}
\label{lemma:avg_edge_size_vs_volume}
    The hypergraph with average hyperedge size $\hat{r}$ cannot have volume larger than $2 \hat{r}^2 n^{\hat{r}}$.
\end{lemma}

These facts combined allow us to conclude Lemma \ref{lemma:convergence_with_avg_r_hypergraph}: in fact when knowing that the volume of the hypergraph is $\leq 2 \hat{r}^2 n^{\hat{r}}\leq 2 \frac{9}{\phi^2} n^{\frac{3}{\phi}}$, and since the mixing time according to Lemma \ref{lemma:fast_convergence} is 
\begin{align}
    t &\leq \frac{\log(\text{vol}(H))}{\phi^2} \\
    & \leq \frac{\log(2 \frac{9}{\phi^2} n^{\frac{3}{\phi}})}{\phi^2} \\
    & \leq O\left(\frac{\log(n)}{\phi^3} + \frac{2\log\left(\frac{1}{\phi}\right)}{\phi^2}\right) && \frac{1}{\phi} \ll n \\
    & \leq O\left(\frac{\log(n)}{\phi^3}\right)
\end{align}

Now it only needs to be proved Lemma \ref{lemma:hyperedge_avg_size_vs_phi} and Lemma \ref{lemma:avg_edge_size_vs_volume}. We start with Lemma     \ref{lemma:hyperedge_avg_size_vs_phi}

\begin{proof}
    The proof of Lemma \ref{lemma:hyperedge_avg_size_vs_phi} is similar to the proof for the $r$-uniform case: in particular using the probabilistic method, we are going to claim that
    
    \begin{equation}
        \mathbb{P}\left(\left\{\phi(S) \leq \frac{3}{\hat{r}}\right\}\right) \geq
    \end{equation}
    \begin{multline}
        \mathbb{P}\left(\left\{\phi(S) \leq \frac{3}{\hat{r}}\right\} \mid \left\{\text{vol}(S) \in \left[\frac{\text{vol}(H)}{3}, \frac{2\text{vol}(H)}{3}\right]\right\}\right) \\ \mathbb{P}\left(\left\{\text{vol}(S)\in\left[\frac{\text{vol}(H)}{3}, \frac{2\text{vol}(H)}{3}\right]\right\}\right)  >  0
    \end{multline}
    
    We are going to say that $\mathbb{P}\left(\left\{\text{vol}(S)\in\left[\frac{\text{vol}(H)}{3}, \frac{2\text{vol}(H)}{3}\right]\right\}\right) > 0$ with this easy argument: assume that there is a node with degree $> \frac{2}{3}\text{vol}(H)$. Then it would be impossible to have a set $S$ with the desired volume: we are claiming that such vertex with very high degree cannot exist. In fact, if $d(v_i) > \frac{2}{3}\text{vol}(H) = \frac{2}{3}m\hat{r} \geq m$ (assuming that $\hat{r}\geq 2$ since all hyperedges should have at least two nodes in them, if we do not allow self-loops). But, no node can have degree strictly larger than the number of edges. If we assume that there is a node with degree $\in\left[\frac{\text{vol}(H)}{3}, \frac{2\text{vol}(H)}{3}\right]$, then we are done (we can achieve a set $S$ with the desired volume by simply adding the single node to it). To conclude if all nodes have degree $\leq \frac{1}{3}\text{vol}(G)$, then it means that by adding nodes one by one to $S$, we cannot overshoot from $\frac{1}{3}\text{vol}(H)$ to $\frac{2}{3}\text{vol}(H)$.
    
    Once we have proven that, it is easy to prove that
    
    \begin{equation}
        \mathbb{P}\left(\left\{\phi(S)\leq \frac{3}{\hat{r}}\right\} \middle| \left\{\text{vol}(S)\in\left[\frac{\text{vol}(H)}{3}, \frac{2\text{vol}(H)}{3}\right]\right\}\right) > 0
    \end{equation}
    
    In fact first we can prove that 
    \begin{equation}
        \mathbb{E}\left[\phi(S) \middle|     \left\{\text{vol}(S)\in\left[\frac{1}{3}\text{vol}(H), \frac{2}{3}\text{vol}(H)\right]\right\}\right] \leq \frac{3}{\hat{r}} 
    \end{equation}
    
    To see why this is true, it is enough to notice that
    
    \begin{align}
        \mathbb{E}\left[\frac{\delta(S, V\setminus S)}{\min(\text{vol}(S), \text{vol}(V\setminus S)} \middle| \left\{\text{vol}(H)\in\left[\frac{1}{3}\text{vol}(H), \frac{2}{3}\text{vol}(H)\right]\right\} \right] & \leq \\
        \frac{\sum_{i=1}^{m} \mathbb{P}\left(\left\{e_i \text{ is cut}\right\} \middle| \left\{\text{vol}(S)\in\left[\frac{1}{3}\text{vol}(H), \frac{2}{3}\text{vol}(H)\right]\right\}\right)}{\frac{1}{3}\text{vol}(H)} & \leq \\ 
        \frac{m}{\frac{1}{3}\hat{r}m} = \frac{3}{\hat{r}}
    \end{align}
    
    Hence, using the fact that 
    \begin{align}
        \mathbb{P}\left(\left\{\phi(S)\leq\frac{3}{\hat{r}}\right\}\middle|\left\{\text{vol}(S)\in\left\{\frac{1}{3}\text{vol}(H), \frac{2}{3}\text{vol}(H)\right\}\right\}\right) \geq
    \end{align}
    \begin{multline}
        \mathbb{P}\biggl(\left\{\phi(S)\leq \mathbb{E}\left[\phi(S)\middle|\left\{\text{vol}(S)\in\left[\frac{1}{3}\text{vol}(H), \frac{2}{3}\text{vol}(H)\right]\right\}\right]\right\} \mid \\ \left\{\text{vol}(S)\in\left[\frac{1}{3}\text{vol}(H), \frac{2}{3}\text{vol}(H)\right]\right\}\biggr) > 0
    \end{multline}
    Which concludes the proof.
\end{proof}

Now, we prove Lemma \ref{lemma:avg_edge_size_vs_volume}:

\begin{proof}
    In order to prove that when the average $\hat{r}$ is small, then the volume of the graph is not larger than $2 \hat{r}^2 n^{\hat{r}}$ it is enough to notice the following fact: 
    \begin{equation}
        \text{vol}(H) = \hat{r} m
    \end{equation}
    In order to estimate how many edges we can have when the average edge size must be $\hat{r}$, we notice the following: in order to maximize $m$, we can take all edges of size $l\leq \hat{r}$, and for every edge of size $l$ we can add for free another edge of size $q = 2\hat{r} - l$ without altering the average edge size $\hat{r}$. (Notice that it is not possible to create more edges than with this technique, without altering the average edge size).
    
    Now, it is enough to count how many edges of size $l\leq \hat{r}$ we can create:
    
    \begin{align}
        \text{vol}(H) & \leq \hat{r} m \\ 
        & \leq 2 \hat{r} \left|\{e\in E \text{ s.t. } |e| \leq \hat{r}\}\right| \\
        & = 2\hat{r} \sum_{l=1}^{
        \hat{r}} {n\choose l} \\ 
        & \leq 2 \hat{r}^2 n^{\hat{r}}
    \end{align}
\end{proof}

With this fact we have just been able to prove that the mixing time for irregular hypergraphs is $O\left(\frac{\log(n)}{\phi^3}\right)$. Although this mixing time is not as good as the one for graphs and regular hypergraphs, notice that whenever the conductance of the graph is $\gg \frac{1}{n}$ (for example, a small constant $O(1)$) than the result found is almost equivalent to Theorem \ref{theorem:mixing_theorem_regular_hypergraphs}, and more importantly ensures an approximation factor for a clustering algorithm based on this mixing result which would be logarithmic in the number of nodes rather than linear.
 
\subsubsection{We can choose the starting vertex wrt the stationary distribution}
\label{subsec:choose_starting_vertex_random}

Another possible solution takes advantage of this fact: it is easy to see that in the proof of Lemma \ref{lemma:fast_convergence} we can easily substitute the definition of $R_0(k):= \sqrt{\frac{\hat{k}}{d(v_0)}}$ with $d(v_0)$ being the degree of the vertex $v_0$ s.t. the initial probability vector is $\chi_{v_0}$. This yields an improved upper bound of $I_t(k) \leq \sqrt{\frac{\hat{k}}{d(v_0)}}e^{-\frac{t \phi^2}{4}} + \frac{k}{\text{vol}(H)}$
We might ask whether we are now able to conclude a mixing time which is logarithmic in the number of nodes $n$: the answer to that is that it clearly depends on how large is $d(v_0)$. In fact if $\text{vol}(G) = 2^n$ and $d(v_0) = O(1)$, then we cannot hope to achieve a better mixing time than $t=O(n)$. But if, instead, $d(v_0) = \text{poly}\left(\frac{1}{n}\right)\text{vol}(G)$, then $\sqrt{\frac{\hat{k}}{\tilde{d}}} = \text{poly}(n)$ and the mixing time becomes once again logarithmic in $n$. 

So the idea is this: if we can pick the starting vertex with probability $\sim \vec{\pi} = \left(\frac{d(i)}{\text{vol}(G)}\right)_{i=1}^{n}$, then we can estimate the probability that we pick a bad (with very little degree) vertex as: 
\begin{align}
    \mathbb{P}\left(\left\{\text{starting vertex has degree} \ll \frac{\text{vol}(G)}{\text{poly}(n)}\right\}\right) & \leq 
    n \cdot \frac{\text{poly}(n)}{\text{vol}(G)} \\ & \xrightarrow[(\text{vol}(G))\to 2^n]{} 0
\end{align}

This is saying that when we are allowed to select the starting vertex with probability proportional to the stationary distribution, then we have a probability close to 1 to select a \textit{good} starting vertex, namely a vertex with high volume which ensures that $\frac{\text{vol}(G)}{\tilde{d}} = O(\text{poly}(n))$, and yields a mixing time $O\left(\frac{\log(n)}{\phi^2}\right)$. Since in the vast majority of probabilistic clustering algorithms, the starting vertex is chosen at random w.p. $\sim \pi$, then we can conclude that our result Theorem \ref{theorem:mixing_theorem} is still powerful enough to provide good mixing guarantees with high probability.

\end{document}